\documentclass[11pt,oneside,a4paper, titlepage]{article}
\addtolength{\voffset}{-15mm}
\usepackage{inputenc}
\usepackage{a4wide}
\usepackage[T1]{fontenc}
\usepackage[english]{babel}
\usepackage{amsmath}
\usepackage{amssymb}
\usepackage{color}
\usepackage{float}
\usepackage{moreverb}
\usepackage{listings}
\usepackage{fancyhdr}
\usepackage{tikz}
\usepackage{dsfont}
\linespread{1.2}
\begin{document}
\hrule height 1pt
\vskip 0.2cm
\noindent \textbf{Name:} Abderrazak \textsc{DERDOURI}\\
\textbf{Object:} CQF Module 2 Examination\\
\textbf{Date}: January 2015 Cohort
\vskip 0.2cm
\hrule height 1pt
\section*{A. Stochastic Calculus [24\%]}
\begin{enumerate}
\item [1.]
\begin{eqnarray*}
dF&=&F(t+dt, S_1+dS_1,...,S_N+dS_N)-F(t, S_1,...,S_N)\\
&=&\frac{\partial F}{\partial t} dt + \sum_{i=1}^{N}\frac{\partial F}{\partial S_i}dS_i +
\frac{1}{2}\sum_{i=1}^{N} \frac{\partial^2 F}{ \partial S_i^{2}} (dS_i)^{2} + \frac{1}{2}\sum_{i\ne j}^{N} \frac{\partial^2 F}{\partial S_i \partial S_j} dS_i dS_j\\
\end{eqnarray*}
since we have: \((dS_i)^{2}=\sigma_i^2S_i^2dt\) and \(dS_id S_j=\rho_{ij}\sigma_i\sigma_jdt\)
\begin{eqnarray*}
dF&=&F(t+dt, S_1+dS_1,...,S_N+dS_N)-F(t, S_1,...,S_N)\\
&=&\frac{\partial F}{\partial t} dt + \sum_{i=1}^{N}\frac{\partial F}{\partial S_i}dS_i +
\frac{1}{2}\sum_{i=1}^{N} \frac{\partial^2 F}{ \partial S_i^{2}} \sigma_i^2S_i^2dt + \frac{1}{2}\sum_{i\ne j}^{N} \frac{\partial^2 F}{\partial S_i \partial S_j} \rho_{ij}\sigma_i\sigma_jdt\\
&=&\frac{\partial F}{\partial t} dt + \sum_{i=1}^{N}\frac{\partial F}{\partial S_i}dS_i +
\frac{1}{2}\bigg[\sum_{i=1}^{N} \frac{\partial^2 F}{ \partial S_i^{2}} \sigma_i^2S_i^2 + \sum_{i\ne j}^{N} \frac{\partial^2 F}{\partial S_i \partial S_j} \rho_{ij}\sigma_i\sigma_j\bigg]dt\\
&=& \bigg[\frac{\partial F}{\partial t} + \frac{1}{2}\sum_{i=1}^{N} \frac{\partial^2 F}{ \partial S_i^{2}} \sigma_i^2S_i^2 + \frac{1}{2}\sum_{i\ne j}^{N} \frac{\partial^2 F}{\partial S_i \partial S_j} \rho_{ij}\sigma_i\sigma_j\bigg]dt + \sum_{i=1}^{N}\frac{\partial F}{\partial S_i}dS_i
\end{eqnarray*}
\item [2.]
\(Y(t)=\exp[\sigma X(t)-\frac{1}{2}\sigma^2 t]\). Let's say: \(Y(t)=f(t, X(t))\).\\
Applying Ito's lemma to \(f(t, X(t))\) we obtain :
\begin{eqnarray*}
df(t, X(t))&=&\frac{\partial f}{\partial t} dt + \frac{\partial f}{\partial X}dX + \frac{1}{2}\frac{\partial^2 f}{\partial X^2} (dX)^2\\
&=&-\frac{1}{2}\sigma^2 \exp\big[\sigma X(t)-\frac{1}{2}\sigma^2 t\big] + \sigma \exp[\sigma X(t)-\frac{1}{2}\sigma^2 t\big]dX + \frac{1}{2} \sigma^2 \exp[\sigma X(t)-\frac{1}{2}\sigma^2 t\big] (dX)^2\\
&=&-\frac{1}{2}\sigma^2 \exp\big[\sigma X(t)-\frac{1}{2}\sigma^2 t\big] + \sigma \exp\big[\sigma X(t)-\frac{1}{2}\sigma^2 t\big]dX + \frac{1}{2} \sigma^2 \exp[\sigma X(t)-\frac{1}{2}\sigma^2 t\big] dt\\
&=&\sigma \exp\big[\sigma X(t)-\frac{1}{2}\sigma^2 t\big]dX\\
&=&Z(t)g(t)dX(t)
\end{eqnarray*}
where \(Z(t)=\exp\big[\sigma X(t)\big]\) and \(g(t)=\sigma\exp\big[-\frac{1}{2}\sigma^2 t\big] \).\\
Because the stochastic process Y(t) is the exponential of another
process and because it is a martingale, Y(t) is in fact 
an exponential martingale.
\item [b)] To check that the process 
\begin{displaymath}
Y(t)=\sqrt{t}X(t)-\int_{0}^{t}\frac{X(s)}{2\sqrt{s}}ds
\end{displaymath}
is a martingale let's check if the drift of 
\(Z(t)=\sqrt{t}X(t)\) is equal to  \(\int_{0}^{t}\frac{X(s)}{2\sqrt{s}}ds\).\\
Let's apply Ito's Lemma to the function \(f(s,x)=\sqrt{s}x\):
\begin{eqnarray*}
df(s, x)&=&\frac{\partial f}{\partial s} ds+\frac{\partial f}{\partial x}dx+\frac{1}{2}\frac{\partial^2 f}{\partial x^2}(dx)^2\\
&=&\frac{1}{2\sqrt{s}} ds + \sqrt{s}dx\\
\end{eqnarray*}
Integrating over [0; t], we have:
\begin{eqnarray*}
Z(t)&=&\int_{0}^{t}\frac{1}{2\sqrt{s}} ds + \int_{0}^{t}\sqrt{t}dX(s)
\end{eqnarray*}
The drift of Z(t) is equal to \(\int_{0}^{t}\frac{1}{2\sqrt{s}}\). Hence 
\(Y(t) =  \int_{0}^{t}\sqrt{t}dX(s)\) and \(Y(t)\) is a martingale.
\end{enumerate}
\section*{B. Portfolio Optimisation [46\%]}
Consider the investment universe composed of the risky assets (A, B, C, D):
\begin{displaymath}
\mathbf{\mu} =
\left( \begin{array}{c}
\mu_A\\
\mu_B\\
\mu_C\\
\mu_D
\end{array} \right)=
\left( \begin{array}{c}
0.04\\
0.08\\
0.12\\
0.15
\end{array} \right),
\end{displaymath}
with a correlation structure:
\begin{displaymath}
\mathbf{R} =
\left( \begin{array}{cccc}
1 & 0.2 & 0.5 & 0.3 \\
0.2 & 1 & 0.7 & 0.4 \\
0.5 & 0.7 & 1 & 0.9 \\
0.3 & 0.4 & 0.9 & 1
\end{array} \right)
\end{displaymath}
and the diagonal standard deviation matrix S:
\begin{displaymath}
\mathbf{S} =
\left( \begin{array}{cccc}
0.07 & 0 & 0 & 0 \\
0 & 0.12 & 0 & 0 \\
0 & 0 & 0.18 & 0 \\
0 & 0 & 0 & 0.26
\end{array} \right),
\end{displaymath}
\begin{enumerate}
\item [1.] Consider the following optimization task for a portfolio where the net of allocations invested (borrowed) in a risk-free asset:
\begin{eqnarray*}
\arg\min_{w} \frac{1}{2}w^{'}\Sigma w \text{ subject to: }  r+(\mu-r\mathds{1})^{'}w=m, \text{where }  m=0.1.
\end{eqnarray*}
The optimisation constraint balances the risk premium for the assets \(\mu-r\) and target return m.\\
There is no need to introduce an extra asset.
\begin{enumerate}
\item [(a)] Obtain analytical solution for optimal portfolio allocations \(w^{*}\). Provide full workings, including derivation and result for the Lagrangian multiplier \(\gamma\).\\
This is an optimization problem with equality constraints. We can solve it
using the method of Lagrange.\\
We form the Lagrange function with a Lagrange multiplier \(\gamma\) 
\begin{eqnarray*}
L(w, \gamma)= \frac{1}{2}w^{'}\Sigma w -\gamma [r+(\mu-r\mathds{1})^{'}w-m]
\end{eqnarray*}
Next, we solve for the first order condition by taking the derivative with respect to the vector w:
\begin{eqnarray*}
\nabla L(w, \gamma)=\frac{\partial L}{\partial \gamma}(w, \gamma)= \Sigma w - \gamma (\mu-r\mathds{1})=0
\end{eqnarray*}
Checking the second order condition: 
\begin{eqnarray*}
HL(w, \gamma)=\frac{\partial ^2L}{\partial \gamma^2}(w, \gamma)= \Sigma
\end{eqnarray*}
the Hessian of the objective function is
equal to the covariance matrix, which is positive definite. Therefore, we have
reached the optimal weight vector \(w^*\):
\begin{eqnarray*}
w^{*}=\Sigma^{-1}*\gamma(\mu-r\mathds{1})
\end{eqnarray*}
or we have :
\begin{eqnarray*}
r+(\mu-r\mathds{1})^{'}w^{*} = m
\end{eqnarray*}
The value of the Lagrangian multiplier is then:
\begin{eqnarray*}
\gamma=\frac{m-r}{(\mu-r\mathds{1})^{'}\Sigma^{-1}(\mu-r\mathds{1})}
\end{eqnarray*}
\item [(b)] For the target return of 10\% and risk-free rate of 3\% calculate optimal allocations \(w^*\) and portfolio standard deviation.\\
The following R program provide elementary calculation to obtain \(w^*\):
\begin{verbatim}
rm(list=ls())   # removes all objects in memory
R = matrix( 
  # the data elements 
  c(1 ,0.2 ,0.5, 0.3 ,0.2 ,1 ,0.7 ,0.4 ,0.5 ,0.7 ,1 ,0.9 ,0.3 ,0.4 ,0.9 ,1),
  nrow=4,              # number of rows 
  ncol=4,              # number of columns 
  byrow = TRUE)        # fill matrix by rows 


S = matrix( 
  # the data elements 
  c(0.07, 0, 0, 0, 0, 0.12, 0, 0, 0, 0, 0.18 ,0, 0, 0, 0, 0.26),
  nrow=4,              # number of rows 
  ncol=4,              # number of columns 
  byrow = TRUE)        # fill matrix by rows 

r=0.03
m=0.1

mu=matrix(
  # the data elements
  c(0.04, 0.08, 0.12, 0.15),
  nrow=4,              # number of rows 
  ncol=1,              # number of columns 
  byrow = TRUE)        # fill matrix by rows 

> (Sigma = S%*%R%*%S)
	0.00490 0.00168 0.00630 0.00546
	0.00168 0.01440 0.01512 0.01248
	0.00630 0.01512 0.03240 0.04212
	0.00546 0.01248 0.04212 0.06760

> (gamma=(m-r)/(t(mu-r)%*%(solve(Sigma))%*%(mu-r)) )
	0.2493658
	
> ( w = ((m-r)*solve(Sigma)%*%(mu-r))/ (t(mu-r)%*%solve(Sigma)%*%(mu-r))[1] )
	0.3957241
	1.0540823
	-0.8268286
	0.7312768
\end{verbatim}

We obtain 
\begin{displaymath}
\mathbf{w^*} =
\left( \begin{array}{c}
0.3957241 \\
1.0540823 \\
-0.8268286 \\
0.7312768 
\end{array} \right)
\end{displaymath}
The following R code give us the portfolio standard deviation:  \(\sigma_{\pi}=\sqrt{{w^*}^{'}\Sigma w^{*}}\) is \begin{verbatim}
> ( sigma_p = sqrt(t(w)%*%Sigma%*%w) )
	0.1321197
\end{verbatim}
\(\sigma_{\pi}=13,22 \%\)
\item [(c)] The tangency portfolio is the portfolio that is entirely invested in risky assets.
For convenience, we define the following scalars:
\begin{displaymath}
\left\{\begin{array}{ll}
A=\mathds{1}^{'}\Sigma^{-1}\mathds{1},\\
B=\mu^{'}\Sigma^{-1}\mathds{1}=\mathds{1}^{'}\Sigma^{-1}\mu\\
C=\mu^{'}\Sigma^{-1}\mu
\end{array} \right.
\end{displaymath}
The tangency portfolio is represented by the point \(\sigma_T, \mu_T\), and the solution to \(\sigma_T\) and \(\mu_T\) are obtained by solving simultaneously:
\begin{displaymath}
\left\{\begin{array}{ll}
\sigma_{T}^2=\frac{A\mu_{T}^2-2B\mu_{T}+C}{\Delta}\\
\mu_{T}=r+\sigma_{T}*\sqrt{C-2rB+r^2A}.\\
\end{array} \right.
\text{with } \Delta=AC-B^2 > 0.
\end{displaymath}
Once \(\mu_T\) is obtained, the corresponding values for \(\gamma_T\) and \(w^{*}\) are:
\(\gamma_T=\frac{\mu_T-r}{C-2rB+r^2A}\) and \(w^{*}=\gamma_{T}\Sigma^{-1}(\mu-r\mathds{1})\).\\
The tangency portfolio is shown to be:\\
\(w^{*}=\frac{\Sigma^{-1}(\mu-r\mathds{1})}{B-Ar}\); \(\mu_{T}=\frac{C-Br}{B-Ar}\); \(\sigma_{T}=\frac{C-2rB+r^2A}{(B-Ar)^{2}}\).
The following R program provide elementary calculation:
\begin{verbatim}
Id=matrix(
  c(1, 1, 1, 1),
  nrow=4,              # number of rows 
  ncol=1,              # number of columns 
  byrow = TRUE)        # fill matrix by rows

Id

> (A = t(Id)%*%solve(Sigma)%*%Id)
	1505.261
> (B = t(mu)%*%solve(Sigma)%*%Id)
	50.58862
> (C = t(mu)%*%solve(Sigma)%*%mu)
	1.961295
> (DELTA=A*C-B*B)
	393.0515
> (wt=solve(Sigma)%*%(mu-r*Id)/drop(B-A*r))
	0.2922081
	0.7783487
	-0.6105414
	0.5399847
\end{verbatim}
\begin{displaymath}
\mathbf{w^*} =
\left( \begin{array}{c}
0.2922081\\
0.7783487\\
-0.6105414\\
0.5399847
\end{array} \right),
\end{displaymath}
The slope of the Capital market line (CML) is the Sharpe ratio of the market portfolio (tangency portfolio):
slope(CML)=\(\frac{E(r_M)-r}{\sigma_M}=\frac{\mu_T-r}{\sigma_T}\)
\begin{verbatim}
> ( sigma_t = sqrt(t(wt)%*%Sigma%*%wt) )
	0.09755895
> (mu_t=r+sigma_t*sqrt(C-2*r*B+r*r*A))
[1,] 0.08168895
> (slope_CML=(mu_t-r)/sigma_t)
	0.5298228
\end{verbatim}
The slope of Capital Market Line is \(0.53\). This indicates how much the expected rate of return must increase if the standard deviation increases by one unit.
\item [2.]
Use tangency portfolio allocations \(w_T\) to estimate the Analytical VaR with c = 99\% confidence:\\
\(VaR(X)=w_{T}^{'}\mu+\sqrt{w_{T}^{'}\Sigma w_{T}}*\text{Factor}\)
\begin{enumerate}
\item [(a)] the Normal distribution Factor = \(\Phi^{-1}(1-c)\).
\begin{verbatim}
> (Factor=qnorm(0.99))
	2.326348
> (VAR = t(wt)%*%mu + sqrt(t(wt)%*%Sigma%*%wt)*Factor)
	0.308645
\end{verbatim}

\item [(b)] the Student's t distribution Factor = \(T_{\nu}^{-1}(1-c)\).
\begin{verbatim}
> (Factor=qnorm(0.99))
	2.326348
> (VAR = t(wt)%*%mu + sqrt(t(wt)%*%Sigma%*%wt)*Factor)
	0.308645
> (Factor=qt(0.99, 30))
	2.457262
> (VAR = t(wt)%*%mu + sqrt(t(wt)%*%Sigma%*%wt)*Factor)
	0.3214168
\end{verbatim}
\end{enumerate}
\end{enumerate}
\end{enumerate}
\end{document}
